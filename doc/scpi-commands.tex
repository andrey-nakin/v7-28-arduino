\documentclass[10pt, a4paper, twocolumn]{article}
\usepackage[utf8]{inputenc}
\usepackage[russian]{babel}
\usepackage{hyperref}
\usepackage[]{graphicx}

\title{Прошивка для Arduino Mega2560, подключённого к вольтметру В7-28}

\newcommand{\SCPI}{\mbox{SCPI}}
\newcommand{\V}{\mbox{В7-28}}
\newcommand{\Arduino}{Arduino Mega}

\newcommand{\CMD}[1]{{\tt #1}}
\newcommand{\PARAM}[1]{"<{\it #1}>"}
\newcommand{\CMDSTUB}[1]{{\tt #1}*}

\newcommand{\SUBSYSTEMSECTION}[1]{\subsection{Подсистема \CMD{#1}}}
\newcommand{\CMDSECTION}[1]{\subsubsection*{\CMD{#1}}}

\begin{document}

\maketitle

\section{Введение}

Данный документ описывает прошивку для микроконтроллера \Arduino или совместимого с ним, которая реализует ограниченный набор \SCPI{} команд цифрового мультиметра. Функции мультиметра выполняет цифровой вольтметр \V.

\subsection{Краткое описание вольтметра}

\V{} --- многофункциональный вольтметр среднего класса точности, выпускается с 1970-х годов по сию пору. Его характеристики уступают современным мультиметрам, таким как Agilent \mbox{34401A}, но для многих применений они достаточны, к тому же ещё много экземпляров \V{} находятся в строю.

\V{} имеет возможность дистанционного управления и снятия показаний, что позволяет его использовать в автоматизированных измерительных установках. Но для управления требуется физически задавать и считывать логические сигналы на разъёмах вольтметра. Разумеется, в самом вольтметре не реализован какой-либо язык управления. В настоящее время для работы с ним как правило используютя аппаратно-программные решения на базе КАМАК.

\subsection{Необходимость данной работы}

Для простоты использования какого-либо устройства в современных измерительных установках требуется наличие у него современных стандартных аппаратного и программного интерфейсов.

В качестве аппаратного интерфейса практически безальтернативно выступает USB, которым сейчас оборудуются даже мобильные телефоны и планшеты, не говоря уже о ноутбуках и стационарных компьютерах.

В качестве программного интерфейса мы видим наиболее подходящей систему команд \SCPI, которая разработа достаточно давно, хорошо стандартизирована и документирована, и используется во многих современных измерительных устройствах.

Таким образом, необходимо <<остастить>> вольтметр \V{} интерфейсом USB и возможностью принимать команды на языке \SCPI. Тогда он с минимальными усилиями (а в идеале~--- сразу и без усилий) сможет войти в состав измерительных установок, где программная часть написана при помощи стандартных средств типа LabVIEW, и используются обычные современные персональные компьютеры без каких-либо плат расширения и модулей КАМАК.

\subsection{Почему Arduino}

\Arduino представляет собой плату микроконтроллера с установленным микропроцессором, программируемой памятью, множеством цифровых портов ввода-вывода и USB-разъёмом. Из всего семейства микроконтроллеров Arduino только Mega обладает числом портов достаточным для полноценного управления \V. Микроконтроллер подключается к компьютеру при помощи USB, а к вольтметру при помощи шлейфа, который легко изготовить самостоятельно.  В память микроконтроллера прошита программа, принимающая команды от компьютера на языке \SCPI{} и переводящая их в управляющие сигналы для \V.

\Arduino является готовым устройством, что удобно для конечного пользователя, не имеющего возможности самостоятельно распаивать платы микроконтроллеров. Он широко распространён, его легко приобрести, по нему доступна обширная документация. Наряду с <<оригинальным>> микроконтроллером, выходящим под торговой маркой Arduino, доступно множество более дешёвых аналогов, которые полностью совместимы с \Arduino при в несколько раз меньшей цене.

\section{Подключение {\Arduino} к \V}

\subsection{Распайка разъёмов}

Управление и считывание данных с {\V} осуществляется по двум разъёмам на задней панели вольтметра: ЦПУ и ДУ.

Распайка разъёмов ЦПУ и ДУ представлена в таблицах~\ref{tab_cpu_wires} (стр.~\pageref{tab_cpu_wires}) и~\ref{tab_rc_wires} (стр.~\pageref{tab_rc_wires}).

\begin{table*}
\begin{center}
\caption{Распайка разъёма ЦПУ (56 контактов)}
\begin{tabular}{cccc}
\hline \hline
Контакт & Контакт & Контакт & Контакт \\
Arduino & разъёма ЦПУ & Arduino & разъёма ЦПУ \\
\hline
GND & 1 & - & А \\
- & 2 & - & Б \\
- & 3 & - & В \\
- & 4 & - & Г \\
- & 5 & - & Д \\
- & 6 & - & Е \\
- & 7 & - & Ж \\
- & 8 & - & З \\
- & 9 & - & И \\
- & 10 & - & К \\
- & 11 & - & Л \\
- & 12 & - & М \\
- & 13 & - & Н \\
- & 14 & - & О \\
- & 15 & - & П \\
- & 16 & - & Р \\
- & 17 & - & С \\
- & 18 & - & Т \\
- & 19 & - & У \\
- & 20 & - & Ф \\
- & 21 & - & Х \\
- & 22 & - & Ц \\
- & 23 & - & Ч \\
- & 24 & - & Ш \\
- & 25 & - & Ы \\
- & 26 & - & Э \\
- & 27 & - & Ю \\
- & 28 & - & Я \\
\hline \hline
\end{tabular}
\label{tab_cpu_wires}
\end{center}
\end{table*}

\begin{table*}
\begin{center}
\caption{Распайка разъёма ДУ (24 контакта)}
\begin{tabular}{cccc}
\hline \hline
Контакт & Контакт & Контакт & Контакт \\
Arduino & разъёма ДУ & Arduino & разъёма ДУ \\
\hline
- & 1 & - & А \\
- & 2 & - & Б \\
- & 3 & - & В \\
- & 4 & - & Г \\
- & 5 & - & Д \\
- & 6 & - & Е \\
- & 7 & - & Ж \\
- & 8 & - & З \\
- & 9 & - & И \\
- & 10 & - & К \\
- & 11 & - & Л \\
- & 12 & - & М \\
\hline \hline
\end{tabular}
\label{tab_rc_wires}
\end{center}
\end{table*}

\section{Список команд}

В данном разделе приведены все \SCPI{} команды, реализованные в прошивке. Порядок изложения --- алфавитный с разбивкой на подсистемы. Команда может сопровождаться комментарием, описывающим особенности реализации.

Данная прошивка реализует ограниченное множество \SCPI{} команд мультиметра Hewlett-Packard (ныне~--- Agilent) \mbox{34401A}. Микроконтроллер подключается к ПЭВМ посредством USB соединения и эмулирует последовательный порт. Таким образом, связка <<микроконтроллер --- вольтметр>> со стороны ПЭВМ выглядит как мультиметр \mbox{34401A}, и может использоваться в измерительных программах без модификации программного кода, хотя и с учётом ограничений \V (по скорости работы, точности и пр.).

Команды, отсутствующие в данном списке, не реализованы. При получении нереализованной или неизвестной команды генерируется ошибка \CMD{-113} <<Undefined header>>.

\SUBSYSTEMSECTION{IEEE-488}

\CMDSECTION{*CLS}
\CMDSECTION{*ESE}
\CMDSECTION{*ESE?}
\CMDSECTION{*ESR?}
\CMDSECTION{*IDN?}

Возвращается строка, имитирующая мультиметр Hewlett-Packard 34401A.

\CMDSECTION{*OPC}
\CMDSECTION{*OPC?}
\CMDSECTION{*RST}
\CMDSECTION{*SRE}
\CMDSECTION{*SRE?}
\CMDSECTION{*STB?}
\CMDSECTION{*TST?}

Из-за отсутствия технической возможности фактически тестирование не производится.

\CMDSECTION{*WAI}

\SUBSYSTEMSECTION{CONFigure}

В командах \CMD{CONFigure:<function>} параметр \PARAM{resolution} не используется, поскольку \V{} не имеет возможности управления точностью измерений.

\CMDSECTION{CONFigure:RESistance}
\CMDSECTION{CONFigure:VOLTage:AC}
\CMDSECTION{CONFigure:VOLTage:DC}
\CMDSECTION{CONFigure:VOLTage:DC:RATio}

\SUBSYSTEMSECTION{INPut}

\CMDSECTION{INPut:IMPedance:AUTO}
\CMDSECTION{INPut:IMPedance:AUTO?}

Команда отрабатывает, состояние сохраняется в оперативной памяти, но реально изменения входного сопротивления \V{} не происходит.

\SUBSYSTEMSECTION{MEASure}

\CMDSECTION{MEASure:RESistance?}
\CMDSECTION{MEASure:VOLTage:AC?}
\CMDSECTION{MEASure:VOLTage:DC?}
\CMDSECTION{MEASure:VOLTage:DC:RATio?}

\SUBSYSTEMSECTION{ROUTe}

\CMDSECTION{ROUTe:TERMinals?}

Всегда возвращает \CMD{FRON}.

\SUBSYSTEMSECTION{SYSTem}

\CMDSECTION{SYSTem:ERRor}
\CMDSECTION{SYSTem:BEEPer}

Команда принимается, но зуммер не звучит в связи с отсутствием соответствующего устройства на плате микроконтроллера.

\CMDSECTION{SYSTem:BEEPer:STATe?}
\CMDSECTION{SYSTem:LOCal}
\CMDSECTION{SYSTem:REMote}

Команда \CMD{SYSTem:REMote} всегда действует аналогично команде \CMD{SYSTem:RWLock}, то есть управление вольтметром с передней панели запрещено в remote-режиме.

\CMDSECTION{SYSTem:RWLock}
\CMDSECTION{SYSTem:VERSion?}

\end{document}
