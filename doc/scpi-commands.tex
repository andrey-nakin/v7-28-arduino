\documentclass[12pt, a4paper]{article}
\usepackage[utf8]{inputenc}
\usepackage[russian]{babel}
\usepackage{hyperref}
\usepackage[]{graphicx}

\title{Прошивка для Arduino Mega2560, подключённого к вольтметру В7-28. \\ Список поддерживаемых SCPI команд}

\newcommand{\CMD}[1]{{\tt #1}}
\newcommand{\CMDSTUB}[1]{{\tt #1}*}

\newcommand{\SUBSYSTEMSECTION}[1]{\subsection{Подсистема \CMD{#1}}}
\newcommand{\CMDSECTION}[1]{\subsubsection*{\CMD{#1}}}

\begin{document}

\maketitle

\section{Введение}

В данном документе приведён список \mbox{SCPI} команд, реализованных в прошивке для микроконтроллера Arduino \mbox{Mega2560}, к которому подключён цифровой вольтметр \mbox{В7-28}.

Данная прошивка реализует ограниченное множество \mbox{SCPI} команд мультиметра Hewlett-Packard (ныне~--- Agilent) \mbox{34401A}. Микроконтроллер подключается к ПЭВМ посредством USB соединения и эмулирует последовательный порт. Таким образом, связка <<микроконтроллер --- вольтметр>> со стороны ПЭВМ выглядит как мультиметр \mbox{34401A}, и может использоваться в измерительных программах без модификации программного кода, хотя и с учётом ограничений \mbox{В7-28} (по скорости работы, точности и пр.).

\section{Список команд}

Команды приведены в алфавитном порядке с разбивкой на подсистемы. Команда может сопровождаться комментарием, описывающим особенности реализации.

Команды, отсутствующие в данном списке, не реализованы. При получении нереализованной или неизвестной команды генерируется ошибка \CMD{-113} <<Undefined header>>.

\SUBSYSTEMSECTION{IEEE-488}

\CMDSECTION{*CLS}
\CMDSECTION{*ESE}
\CMDSECTION{*ESE?}
\CMDSECTION{*ESR?}
\CMDSECTION{*IDN?}

Возвращается строка, имитирующая мультиметр Hewlett-Packard 34401A.

\CMDSECTION{*OPC}
\CMDSECTION{*OPC?}
\CMDSECTION{*RST}
\CMDSECTION{*SRE}
\CMDSECTION{*SRE?}
\CMDSECTION{*STB?}
\CMDSECTION{*TST?}

Из-за отсутствия технической возможности фактически тестирование не производится.

\CMDSECTION{*WAI}

\SUBSYSTEMSECTION{MEASure}

\CMDSECTION{MEASure:VOLTage:DC?}
\CMDSECTION{MEASure:VOLTage:DC:RATio?}
\CMDSECTION{MEASure:VOLTage:AC?}
\CMDSECTION{MEASure:RESistance?}

\SUBSYSTEMSECTION{ROUTe}

\CMDSECTION{ROUTe:TERMinals?}

Всегда возвращает \CMD{FRON}.

\SUBSYSTEMSECTION{SYSTem}

\CMDSECTION{SYSTem:ERRor}
\CMDSECTION{SYSTem:BEEPer}

Команда принимается, но зуммер не звучит в связи с отсутствием соответствующего устройства на плате микроконтроллера.

\CMDSECTION{SYSTem:BEEPer:STATe?}
\CMDSECTION{SYSTem:LOCal}
\CMDSECTION{SYSTem:REMote}

Команда \CMD{SYSTem:REMote} всегда действует аналогично команде \CMD{SYSTem:RWLock}, то есть управление вольтметром с передней панели запрещено в remote-режиме.

\CMDSECTION{SYSTem:RWLock}
\CMDSECTION{SYSTem:VERSion?}

\end{document}
